\documentclass[a4paper]{article}
% generated by Docutils <http://docutils.sourceforge.net/>
\usepackage{fixltx2e} % LaTeX patches, \textsubscript
\usepackage{cmap} % fix search and cut-and-paste in Acrobat
\usepackage{ifthen}
\usepackage[T1]{fontenc}
\usepackage[utf8]{inputenc}

%%% Custom LaTeX preamble
% PDF Standard Fonts
\usepackage{mathptmx} % Times
\usepackage[scaled=.90]{helvet}
\usepackage{courier}

%%% User specified packages and stylesheets

%%% Fallback definitions for Docutils-specific commands

% hyperlinks:
\ifthenelse{\isundefined{\hypersetup}}{
  \usepackage[colorlinks=true,linkcolor=blue,urlcolor=blue]{hyperref}
  \urlstyle{same} % normal text font (alternatives: tt, rm, sf)
}{}
\hypersetup{
  pdftitle={Learn Linux the Hard Way},
}

%%% Title Data
\title{\phantomsection%
  Learn Linux the Hard Way%
  \label{learn-linux-the-hard-way}}
\author{}
\date{}

%%% Body
\begin{document}
\maketitle
%
\begin{itemize}

\item Hast Du Angst vor weißer Schrift auf schwarzem Grund?

\item Dir fehlen zum ultimativen Nerdsein nur noch die Skillz auf der  Kommandozeile?

\item Hast du dir schon jahrelang vorgenommen, in die Linux-Welt einzutauchen, aber nie genug Motivation aufbringen können?

\end{itemize}

Dann nimm doch am Seminar ``Learn Linux the Hard Way'' teil und erhalte obendrein noch einen Schein dazu.

Wir werden gemeinsam ein Linux-System aufsetzen und dieses dann für verschiedene Anwendungsgebiete (Webserver, NAS, Desktop, Scientific Computing) individualisieren.

Ziel der Veranstaltung ist es, fundamentales Wissen über die Interna und universelle Bedienkonzepte einer modernen Linux-Distribution kennen zu lernen und diese selbständig zur Lösung von alltäglichen und wiederkehrenden Problemen einzusetzen.

Eine Vorbesprechung findet

Mittwoch, 25.07.2012, 14:00 Uhr, Raum xx-xxx

Wer an der Vorsprechung verhindert ist, kann sich gerne auch per eMail anmelden: \href{mailto:hundt@uni-mainz.de}{hundt@uni-mainz.de}

\end{document}
