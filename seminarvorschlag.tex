%Dokumententyp
\documentclass[a4paper]{scrartcl}
%Zeichenkodierung und Spracheinstellungen
\usepackage[T1]{fontenc}
\usepackage[utf8]{inputenc}
\usepackage{ngerman}
\usepackage[ngerman]{babel}
\usepackage[babel]{csquotes}
\usepackage{url}

%Graphikunterstützung
\usepackage{graphicx}

%Schriftarten
\usepackage{MyriadPro}
\usepackage{MinionPro}

%Kopf- und Fusszeilen
%\usepackage{scrpage2}
%\pagestyle{scrheadings}

%\ihead{obenlinks}
%\chead{oben}
%\ohead{obenrechts}
%\ifoot{untenlinks}
%\cfoot{unten}
%\ofoot{untenrechts}

\date{}
\author{Name}
\title{Seminarvorschlag}

\begin{document}

\maketitle

\begin{abstract}
\textbf{Gesucht:} Eine Seminarveranstaltung im Bachelorstudiengang Informatik der 
Johannes Gutenberg-Universität Mainz, die einen breiteren Interessenbereich
ansprechen kann.
\end{abstract}

\section{Grundidee}

Meine Beobachtung ist, dass nur ein geringer Anteil der Informatikstudenten
tiefere Kenntnisse von Linux-Systemen und freier Software im allgemeinen besitzt.

Ich denke, durch ein Seminar, in dem eine sinnvolle Mischung aus professioneller
Anleitung und selbst erarbeiteter Vorträge, in dem auch detaillierter auf
Hintergründe (auch rechtlicher, politischer, soziologischer Art) eingegangen 
wird könnte sowohl die Akzeptanz von GNU/Linux als Betriebssystem steigern,
als auch durch die Vermittlung von Detailwissen den Forschungsdrang der
Studierenden anregen.

\section{Namensvorschläge}

\begin{itemize}
\item Freie Software in der Anwendung
\item Alternative Betriebssysteme
\item GNU/Linux et al.
\end{itemize}

\section{Lernziele}

\begin{itemize}
\item Lizenzbezogene Besonderheiten von freier Software

\item Erstellen eigener Binaries
\begin{itemize}
\item Kernel-Konfiguration
\item Installation eines Linux from Scratch (\url{http://www.linuxfromscratch.org/}) (das bekommt keiner hin, besser gentoo)
\end{itemize}

\item Bash-Skripting

\item Verständnis von Funktionsweisen und Entwicklungsprozessen
\begin{itemize}
\item Kernel-Architektur (Vgl. Linux <-> BSD)
\end{itemize}

\item \LaTeX{}
\item git als besseres subversion Substitut

\item LPIC-Inhalte (\url{http://www.lpi.org/eng/certification/the_lpic_program/lpic_1})

\end{itemize}

\end{document}
